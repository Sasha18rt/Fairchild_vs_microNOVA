\documentclass[a4paper,12pt]{article}

\usepackage[utf8]{inputenc}
\usepackage[english]{babel}
\usepackage{graphicx}
\usepackage{hyperref}
\usepackage{amsmath}
\usepackage{listings}
\usepackage{geometry}
\usepackage{fancyhdr}
\geometry{margin=1in}
\pagestyle{fancy}
\fancyhf{}
\fancyhead[L]{Fairchild 9440 vs. microNOVA MP/100}
\fancyfoot[C]{\thepage}

\title{Comparison of Fairchild 9440 and microNOVA MP/100 Architectures}
\author{Oleksandr Rotaienko}
\date{\today}

\begin{document}

\maketitle

\tableofcontents
\newpage

\section{Introduction}

The purpose of this report is to compare two computer architectures, Fairchild 9440 and microNOVA MP/100, both of which were launched in the same technological era. This comparison will provide insights into their design philosophies, capabilities, and use cases.

The Fairchild 9440 and microNOVA MP/100 were developed during a time when transistor-based computers were becoming more common, marking the transition from low-scale integration (LSI) to larger-scale integration in computing. By analyzing these architectures, we can understand the constraints and innovations of the time.


\section*{References}
\begin{itemize}
    \item Fairchild 9440 Datasheet. Available at: \url{https://archive.org/details/bitsavers_fairchildmflameDataSheetDec78_2098325/page/n5/mode/2up}
    \item microNOVA MP/100 Technical Documentation. Available at: \url{https://datageneral.uk/restorations/mpt-100-micronova/}
\end{itemize}

\end{document}
