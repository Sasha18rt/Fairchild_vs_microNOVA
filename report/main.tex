\documentclass[a4paper,12pt]{article}
\usepackage{float}

\usepackage[utf8]{inputenc}
\usepackage[english]{babel}
\usepackage{graphicx}
\usepackage{hyperref}
\usepackage{amsmath}
\usepackage{listings}
\usepackage{geometry}
\usepackage{fancyhdr}
\geometry{margin=1in}
\pagestyle{fancy}
\fancyhf{}
\fancyhead[L]{Fairchild 9440 vs. microNOVA MP/100}
\fancyfoot[C]{\thepage}

\title{Comparison of Fairchild 9440 and microNOVA MP/100 Architectures}
\author{Oleksandr Rotaienko}
\date{\today}

\begin{document}

\maketitle

\tableofcontents
\newpage

\section{Introduction}

The purpose of this report is to compare two computer architectures, Fairchild 9440 and microNOVA MP/100, both of which were launched in the same technological era. This comparison will provide insights into their design philosophies, capabilities, and use cases.

The Fairchild 9440 and microNOVA MP/100 were developed during a time when transistor-based computers were becoming more common, marking the transition from low-scale integration (LSI) to larger-scale integration in computing. By analyzing these architectures, we can understand the constraints and innovations of the time.

\section{Elementary Base of Architectures}

\subsection{Fairchild 9440}

\subsubsection{Elementary Base}
\begin{itemize}
    \item Utilizes bipolar integrated circuits with Isoplanar Integrated Injection Logic (I$^2$L) technology.
    \item Based on Low Scale Integration (LSI), focusing on efficient logic and reduced transistor count for compact processing.
\end{itemize}

\subsubsection{Physical Characteristics}
\begin{itemize}
    \item Packaged in a 40-pin DIP (Dual In-line Package).
    \item Operates with a 5V power supply and consumes approximately 1W of power.
    \item Supports compact, high-speed operations suitable for minicomputer-class systems.
\end{itemize}

\subsection{microNOVA MPT/100}

\subsubsection{Elementary Base}
\begin{itemize}
    \item Hybrid system using multiple integrated circuits, such as mN602 and mN615 (proprietary Data General chips), along with additional chips like Intel 8048-based processors for peripheral functions.
    \item DRAMs (4116N) for memory, along with UARTs (2651) for communication.
    \item Leverages technologies from the era of LSI, with some early use of VLSI for specific functions.
\end{itemize}

\subsubsection{Physical Characteristics}
\begin{itemize}
    \item Desktop computer format with modular components.
    \item Includes components like a CRT display and floppy disk drives, adding bulk.
    \item Moderate power consumption, given the mix of ICs and memory units.
\end{itemize}

Both architectures represent the transition from small-scale to more integrated processing units and reflect technological limitations like reliance on 5V logic, discrete memory units, and modular designs typical of the late 1970s.

\section{Architecture Type Comparison}

\subsection{Fairchild 9440}
The Fairchild 9440 architecture is accumulator-based. It features four 16-bit accumulators (AC0-AC3), which act as the primary storage for operands in arithmetic and logical operations. These accumulators are general-purpose and play a significant role in data movement and calculations, making it an accumulator-based architecture.

\subsection{microNOVA MP/100}
The microNOVA MP/100 architecture is also accumulator-based. It relies heavily on accumulators for data processing and operand storage. The specifics of the accumulator implementation are detailed in its technical documentation.

\subsection{Comparison}
Both systems are accumulator-based, leveraging accumulators as central components for their operations.

\section{Addressing Mechanisms}


\subsection{Fairchild 9440}
The Fairchild 9440 processor uses a two-address architecture, where most instructions specify a source operand and a destination operand. This means the same instruction can perform operations and store the result in a different location or overwrite the source.
\subsection{microNOVA MP/100}
microNOVA MP/100: The microNOVA MP/100 is a one-address architecture machine. Most instructions operate with a single explicit memory address operand, using the accumulator implicitly as the source or destination for operations

\subsection{Comparison}
Thus, the Fairchild 9440 is a two-address machine, while the microNOVA MP/100 operates as a one-address machine.

\section{Registers in Fairchild 9440 and microNOVA MP/100}

The register configurations of the Fairchild 9440 and microNOVA MP/100 architectures differed significantly in terms of type, count, and functionality.

\subsection{Fairchild 9440}

The Fairchild 9440 featured a \textbf{set of general-purpose registers} that supported its three-address instruction format. These registers were used for data manipulation, arithmetic operations, and temporary storage of intermediate results. The architecture had \textbf{16 general-purpose registers}, each with a width of \textbf{16 bits}. These registers provided a flexible environment for handling operations, minimizing memory access, and optimizing computational performance.

Additionally, the Fairchild 9440 included specialized registers, such as:
\begin{itemize}
    \item \textbf{Program Counter (PC):} Used to store the address of the next instruction to be executed.
    \item \textbf{Status Register:} Contained flags for arithmetic and logic operations, such as carry, zero, overflow, and sign.
\end{itemize}

The combination of general-purpose and specialized registers enabled efficient handling of both simple and complex instructions.

\subsection{microNOVA MP/100}

The microNOVA MP/100, on the other hand, relied heavily on an \textbf{accumulator-based architecture}. The accumulator served as the primary working register for most arithmetic and logical operations. In addition to the accumulator, the architecture included the following specialized registers:
\begin{itemize}
    \item \textbf{Instruction Register (IR):} Held the current instruction being executed.
    \item \textbf{Program Counter (PC):} Tracked the memory address of the next instruction.
    \item \textbf{Stack Pointer (SP):} Used for managing the call stack during subroutine calls and interrupts.
\end{itemize}

Unlike the Fairchild 9440, the microNOVA MP/100 had fewer general-purpose registers, which limited its ability to store intermediate values and increased dependency on memory access. The accumulator and specialized registers in the microNOVA MP/100 were \textbf{8 bits wide}, reflecting the simpler and more compact design of this architecture.

\subsection{Comparison}

\begin{table}[H]
\centering
\resizebox{\textwidth}{!}{%
\begin{tabular}{|l|c|c|}
\hline
\textbf{Feature} & \textbf{Fairchild 9440} & \textbf{microNOVA MP/100} \\ \hline
General-purpose registers & 16 (16 bits each) & None \\ \hline
Specialized registers & Program Counter, Status & Program Counter, Stack Pointer, Instruction Register \\ \hline
Accumulator width & Not used & 8 bits \\ \hline
\end{tabular}%
}
\caption{Comparison of Fairchild 9440 and microNOVA MP/100 Registers}
\end{table}


In conclusion, the Fairchild 9440's extensive set of general-purpose registers offered flexibility and performance advantages, while the microNOVA MP/100 relied on a simpler, accumulator-based design, which was resource-efficient but less versatile for complex computations.


\section*{References}

\begin{itemize}
    \item Fairchild 9440 Datasheet. Available at: \url{https://archive.org/details/bitsavers_fairchildmflameDataSheetDec78_2098325/page/n5/mode/2up}
    \item microNOVA MP/100 Technical Documentation. Available at: \url{https://datageneral.uk/restorations/mpt-100-micronova/}
\end{itemize}

\end{document}
